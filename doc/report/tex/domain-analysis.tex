\section{Analisi del Dominio}

Il progetto ha lo scopo di simulare il ciclo di vita di creature virtuali denominate \textit{blob} e immerse in un ambiente virtuale 2D dotato di fonti di cibo ed ostacoli.
Prima di ogni simulazione sarà possibile tramite un'apposita UI permettere la parametrizzazione di diverse caratteristiche del mondo simulato da parte dell'utente.
Le caratteristiche ambientali che possono essere parametrizzate sono:
\begin{itemize}
    \item numero di entità Blob presenti nella simulazione. I blob generati all'inizio della simulazione si dividono in:
    \begin{enumerate}
        \item Base Blob;
        \item Cannibal Blob.
    \end{enumerate}
    \item numero di entità Plant presenti all'interno della simulazione. I tipi di piante si suddividono in:
    \begin{enumerate}
        \item Standard Plant;
        \item Reproducing Plant;
        \item Poisonous Plant.
    \end{enumerate}
    \item numero di entità Obstacle presenti all'interno della simulazione;
    \item la luminosità all'interno del mondo simulato;
    \item la temperatura all'interno del mondo simulato;
    \item la durata in giorni della simulazione.
\end{itemize}

Una volta avviata la simulazione verranno disposte le entità all'interno del mondo, e i blob inizieranno quindi a muoversi in cerca di cibo, cercando di non morire in quanto la loro vita si riduce nel tempo. 

Un blob può avere diversi valori di dimensione, velocità e ampiezza del campo visivo, che definiscono il progresso della simulazione.

Vi sono presenti diverse tipologie di blob:
\begin{itemize}
    \item Base Blob: blob standard il quale obiettivo è quello di sopravvivere il più a lungo possibile, consumando il cibo che trova all'interno del mondo. Si muoverà in direzione del cibo più vicino nel caso sia all'interno del suo campo visivo;
    \item Cannibal Blob: diversamente dal Base Blob questo tipo di blob può nutrirsi di cibo e degli altri Base Blob, se questi ultimi hanno una dimensione minore;
    \item Poison Blob: il Poison Blob è una tipologia di blob con un effetto temporaneo applicato da un cibo velenoso. La sua vita si riduce in maniera più rapida per un certo periodo di tempo;
    \item Slow Blob: lo Slow Blob è una tipologia di blob con un effetto temporaneo applicato da un ostacolo. La sua velocità è ridotta per un certo periodo di tempo.
\end{itemize}

Il cibo viene prodotto ciclicamente da un pianta, ogni pianta produce una certa tipologia di cibo, ed ogni cibo può avere diversi \textit{effetti} sui blob:
\begin{itemize}
    \item Standard Effect: il cibo incrementa la vita del blob che lo ha mangiato;
    \item Poisonous Effect: il cibo avvelena il blob facendolo diventare un PoisonBlob;
    \item Reproduce Effect: il cibo permette al blob di riprodursi creando un Blob figlio differente dal padre.
\end{itemize}
Dopo un determinato numero di iterazioni a partire dalla sua generazione o alla collisione con un blob, il cibo viene rimosso dal mondo.

Il blob prima di muoversi verso un cibo deve prima poterlo vedere nel suo campo visivo. Il campo visivo di un blob dipende da un valore iniziale e dalla luminosità che varia durante le fasi di un giorno. Intuitivamente quando è più luminoso i blob potranno vedere ad una distanza maggiore rispetto a quando la luminosità si abbassa.

Così come la luminosità anche la temperatura ha un'effetto attivo sui blob, che anch'essa varia durante l'arco della giornata influenzando la velocità di un blob.

Un blob muovendosi può incorrere in un ostacolo, e ogni ostacolo ha un effetto diverso:
\begin{enumerate}
    \item Slow Effect: l'ostacolo rallenta il blob facendolo diventare Slow Blob;
    \item Damage Effect: l'ostacolo danneggia il blob causandogli danno.
\end{enumerate}