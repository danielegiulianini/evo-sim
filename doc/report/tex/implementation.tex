\section{Implementazione}
Il seguente capitolo motiva e dettaglia le scelte implementative ritenute rilevanti per una corretta comprensione del progetto. È bene precisare che nel codice è presente la Scaladoc che permette di dettagliare ogni singola parte del codice.


\subsection{Paradigma funzionale}
Sin dalle prime fasi di progettazione il team ha intrapreso la scelta di utilizzare il più possibile il paradigma funzionale, cercando di non ricorrere alle usuali abitudini di programmazione object-oriented. Per fare ciò sono stati utilizzati diversi criteri: 
\begin{itemize}
    \item inutilizzo dei side effect, creando ad ogni modifica un nuovo ogetto immutabile.
    \item utilizzo di funzioni ricorsive.
    \item utilizzo di funzioni higher-order che permettono una facile ed immediata realizzazione del parttern Strategy, consentono una maggiore riutilizzabilità del codice. In questo modo è possibile passare alle funzioni strategie esterne, non necessitando così di modificare il codice.
\end{itemize}

Rispettando quanto sopra elencato, il Controller e il Model dell'applicazione sono stati realizzati con un approccio puramente funzionale. La View adottare anch'essa il paradigma descritto in questo corso attraverso l'utilizzo della libreria cats.effect.IO.

\subsubsection{View}
Bisogna parlare della View, in particolare dell'approccio funzionale utilizzato.


\subsubsection{Monadi}


\subsubsection{World e Stream}
Il World è stato modellato come un contenitore immutabile delle proprietà della simulazione e delle entità che vi partecipano. Al fine di memorizzare i dati necessari ai fini di analisi, esso è definito in termini di se stesso: la worldHistory è uno stream costituito dalle istanze di World riferite alle iterazioni precedenti. La lazy evaluation degli stream permette di non appesantire la computazione del simulationLoop con un carico di lavoro altrimenti insostenibile: impiegare una collezione non lazy come una lista avrebbe comportato la creazione di una moltitudine di strutture dati intermedie di grandissime dimensioni. 

\begin{figure}[h!]
\centering
\includegraphics[scale=0.30]{img/WorldDetail.png}
\caption{Dettaglio di worldHistory come Stream per la memorizzazione delle informazioni sugli World alle iterazioni passate necessarie alla elaborazione statistica finale}
\label{fig:worldDetail}
\end{figure}

Grazie agli stream, la valutazione della grande quantità di dati accumulata avviene solo alla conclusione della simulazione e al momento del processamento dei dati a fini statistici, non interferendo così con l’andamento della simulazione.

\subsubsection{Currying e partial application}


\subsubsection{Memoizing}

\subsection{Paradigma logico}
Dopo un accurata e dettagliata analisi della logica di movimento delle entità, il team si era proposto di integrare nel progetto il paradigma logico. Ciò è avvenuto attraverso la libreria TuProlog. In particolare, la logica di movimento ideata e implementata è la seguente:
\begin{itemize}
    \item nel caso in cui nessuna entità commestibile sia presente all'interno del campo visivo, l'entità in considerazione deve muoversi in una direzione casuale.
    \item nel caso in cui una o più entità commestibili siano presenti all'interno del campo visivo, l'entità dovrà dirigersi verso il nutrimento più vicino.
\end{itemize}
La teoria Prolog presente nel file \code{src/main/resources/movementTheory.pl} consente di ricercare la presenza di un entità commestibile all'interno del campo visivo tra tutte le entità presenti nella simulazione, ed agire di conseguenza nel caso in cui l'entità sia presente o meno all'interno del campo visivo. Tuttavia per problemi legati alle performance (in quanto non possibile rispettare il requisito 4.1) si è adottato un approccio ibrido, in cui la ricerca di un entità all'interno del campo visivo viene effettuata con il paradigma funzionale, lasciando poi il calcolo della nuova posizione al paradigma logico.

\subsection{Test}
La realizzazione di test è stato un aspetto fondamentale per verificare la correttezza e la qualità del codice, garantendo anche un'ulteriore documentazione del software sviluppato. La loro implementazione non è avvenuta con la metodologia stringente del Test Driven Development (TDD), bensì con un approccio più tradizionale, in cui la funzionalità veniva testata una volta pronta. In particolare è stata utilizzata la libreria ScalaTest, mantenendo uno stile omogeneo utilizzando \textbf{FunSpec}.

\subsection{Suddivisione del lavoro}
Inizialmente il team ha effettuato un analisi preliminare, dettagliando quali caratteristiche e requisti il software dovesse esibire. In seguito a ciò, si sono susseguiti una frequente serie di meeting (in cui era presente tutto il team) dove, partendo da piccoli diagrammi UML che descrivevano a grandi linee l'architettura del sistema, si è scesi sempre più in dettaglio fino a creare l'intero design architetturale. È bene precisare che il design architetturale è stato ideato in ciascuna sua parte da tutto il team, in modo coeso e collaborativo. Successivamente, in seguito alla realizzazione di un design architetturale dettagliato per ogni funzionalità del software, a ciascun membro del team è stata assegnata una parte dello sviluppo. Nel dettaglio:

\subsubsection{Rei Beshiri}
Il mio ruolo nel progetto dal punto di vista implementativo riguarda principalmente lo sviluppo del model e delle sue diverse componenti in collaborazione con i membri del team, in particolare delle entità \textit{blob} del loro comportamento e reazione all'ambiente circostante e alle intersezioni tra le varie entità in gioco e degli effetti delle stesse così come della loro bounding box.

I test sviluppati riguardano le classi \code{BlobTest}, \code{DegradationTest}, \code{IntersectionTest}.

\subsubsection{Andrea Betti}
Ho contribuito insieme a Rei Beshiri all'implementazione dei comportamenti delle diverse entità della simulazione in \code{evo\_sim.model.EntityBehaviour} e agli effetti applicabili dalle entità \textit{Effectful} in \code{evo\_sim.model.effects.CollisionEffect}, realizzando le classi e le funzioni relative alle entità \textit{Food} e \textit{Plant} e contribuendo in misura minore ai comportamenti delle entità \textit{Blob} e all'estensione di \code{evo\_sim.model.EntityStructure}.

Ho inoltre implementato la logica di rappresentazione della simulazione utilizzando \texttt{scala-swing} all'interno della classe \code{evo\_sim.view.swing.custom.components.ShapesPanel}, utilizzata nella funzione \code{rendered} di \code{evo\_sim.view.swing.SwingView} realizzata da Alessandro Oliva.

Per quanto riguarda i test, ho realizzato \code{FoodTests} e \code{PlantTests}.

\subsubsection{Daniele Giulianini}

\subsubsection{Alessandro Oliva}
Il mio contributo nel model è consistito nello sviluppo delle entità \textit{Obstacle}, le entità con status temporanei (il cui raffinamento è però stato curato da Rei Beshiri), in particolare lo \textit{SlowBlob}.

Mi sono occupato inoltre del ciclo giorno notte della simulazione, con rispettiva influenza sui blob in termini di velocità e campo visivo tramite appositi moduli di funzioni.

Per quanto riguarda la View, ne ho seguito lo sviluppo dalla prima versione in \texttt{ScalaFX} fino all'implementazione puramente funzionale attraverso il framework \texttt{Cats}, del quale con l'importante contributo di Daniele Giulianini è stato sviluppato un intero package che permette di utilizzare componenti Swing in maniera funzionale. Sempre con Daniele Giulianini mi sono occupato dell'integrazione fra View e Core mediante il framework monadico. Parallelamente a questa versione è stata sviluppata un interfaccia a linea di comando per permettere la fruizione dell'applicazione nella sua interezza anche mentre si lavorava alla View.

I test da me sviluppati sono inclusi in \code{ObstacleTests}.
\subsubsection{Andrea Vaienti}